\documentclass{article}
\usepackage{graphicx} % Required for inserting images
\setlength\parindent{0pt}
\usepackage{booktabs}
\usepackage{amsmath}
\usepackage{amssymb}

\title{Fiona's Problem}
\author{SimTech Gang}
\date{March 2025}

\begin{document}

\maketitle

\section{Problem}
God picks two numbers $a,b \in \{2,\ldots,99\}$ and gives their product $P = a\cdot b$ to Gauss and their sum $S = a+b$ to Euler. The two mathematicians are not allowed to communicate, but they can speak out loud their thoughts.

The conversation proceeds as follows:
\begin{enumerate}
\item Gauss: \emph{I don't know the numbers.}
\item Euler: \emph{I knew that you didn't know.}
\item Gauss: \emph{Now I know the numbers.}
\item Euler: \emph{Now I know them too.}
\end{enumerate}

The problem now is to find out what the numbers $a$ and $b$ are and how Gauss and Euler were able to find out.

\section{Solution}
Gauss states he does not know the solution. This means that the factorization of $P$ into two numbers $a,b$ with $a \cdot b = P$ is not unique.

Now, Euler gives the hint that he \emph{knew}, that Gauss cannot know. This indicates that for every possible pair of numbers $a',b'$ that sum up to $S = a' + b'$, the product $P' = a'\cdot b'$ cannot be factorized uniquely into two factors $a$ and $b$.


By going through all possible sums $S \in \{4,\ldots,198\}$, we find the following values for $S$ that fulfill this property:

11, 17, 23, 27, 29, 35, 37, 41, 47, 53

and Euler has to have one of these numbers.

Gauss knows this and can now determine the numbers $a$ and $b$. To understand how this works, we can look at all possible sums Euler could have been told, shown in Table \ref{full dubioses dict}.

\begin{table}[h!]
\centering
\begin{tabular}{ll}
\toprule
\textbf{sum} (Euler) & \textbf{possible products} (Gauss) \\ \toprule
11 & 18, 24, 28, 30 \\ \midrule
17 & 30, 42, 52, 60, 66, 70, 72 \\ \midrule
23 & 42, 60, 76, 90, 102, 112, 120, 126, 130, 132 \\ \midrule
27 & 50, 72, 92, 110, 126, 140, 152, 162, 170, 176, 180, 182 \\ \midrule
29 & 54, 78, 100, 120, 138, 154, 168, 180, 190, 198, 204, 208, 210 \\ \midrule
35 & 66, 96, 124, 150, 174, 196, 216, 234, 250, 264, 276, 286, 294, \\ & 300, 304, 306 \\ \midrule
37 & 70, 102, 132, 160, 186, 210, 232, 252, 270, 286, 300, 312, 322, \\ & 330, 336, 340, 342 \\ \midrule
41 & 78, 114, 148, 180, 210, 238, 264, 288, 310, 330, 348, 364, 378, \\ & 390, 400, 408, 414, 418, 420 \\ \midrule
47 & 90, 132, 172, 210, 246, 280, 312, 342, 370, 396, 420, 442, 462, \\ & 480, 496, 510, 522, 532, 540, 546, 550, 552 \\ \midrule
53 & 102, 150, 196, 240, 282, 322, 360, 396, 430, 462, 492, 520, 546, \\ & 570, 592, 612, 630, 646, 660, 672, 682, 690, 696, 700, 702 \\ \bottomrule
\end{tabular}

\caption{All possible sums and the corresponding products that Euler can have.}
\label{full dubioses dict}

\end{table}

\newpage

From this, we cannot find out how Gauss knows the values of $a,b$. By taking a closer look at Table \ref{full dubioses dict} we can see that possible products show up more than once as they can be formed for different sums. For example, the number 30 can be found for $S = 11$ and $S = 17$. The values for $a,b$ are then:

\begin{align*}
    a = 6, b = 5 \hspace{10pt} \Rightarrow \hspace{10pt} S = 11, P = 30, \\
    a = 15, b = 2 \hspace{10pt} \Rightarrow \hspace{10pt} S = 17, P = 30.
\end{align*}

From this, we can conclude that Gauss cannot have numbers that appear multiple times, like the 30, because he would not be able to tell the numbers $a,b$ in this case.

By removing all numbers that occur multiple times, we receive Table \ref{edited dubioses dict}.

\newpage 

\begin{table}[h!]
\centering
\begin{tabular}{ll}
\toprule
\textbf{sum} (Euler) & \textbf{possible products} (Gauss) \\ \toprule
11 & 18, 24, 28 \\ \midrule
17 & 52 \\ \midrule
23 & 76, 112, 130 \\ \midrule
27 & 50, 92, 110, 140, 152, 162, 170, 176, 182 \\ \midrule
29 & 54, 100, 138, 154, 168, 190, 198, 204, 208 \\ \midrule
35 & 96, 124, 174, 216, 234, 250, 276, 294, 304, 306 \\ \midrule
37 & 160, 186, 232, 252, 270, 336, 340 \\ \midrule
41 & 114, 148, 238, 288, 310, 348, 364, 378, 390, 400, 408, 414, 418 \\ \midrule
47 & 172, 246, 280, 370, 442, 480, 496, 510, 522, 532, 540, 550, 552 \\ \midrule
53 & 240, 282, 360, 430, 492, 520, 570, 592, 612, 630, 646, 660, 672, \\ & 682, 690, 696, 700, 702 \\ \bottomrule
\end{tabular}

\caption{Table \ref{full dubioses dict} after removing numbers that appear more than once.}
\label{edited dubioses dict}

\end{table}

These are all possible products that Gauss can have and uniquely determine the values of $a,b$. Gauss now knows $a,b$ and tells Euler that he does. Euler answers, that he now knows too.

By looking at Table \ref{edited dubioses dict}, we can see why this works. If Gauss would have the number 18, he would be able to conclude that Euler has the number 11, and could determine $a $and $b$. But Euler would not be able to tell, because he does not know whether Gauss has 18, 24, or 28.

So, the only possibility left is the second row in Table \ref{edited dubioses dict}, where Euler also knows both the sum and the product.

With this, we can conclude that Euler has the number $S = 17$, and Gauss $P = 52$.

This uniquely determines the numbers $a = 13$ and $b = 4$. $\square$

\end{document}


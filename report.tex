\documentclass{article}
\usepackage{graphicx} % Required for inserting images

\title{Fiona's Problem}
\author{SimTech Gang}
\date{March 2025}

\begin{document}

\maketitle

\section{Introduction}
God picks two numbers $a,b \in \{2,\ldots,99\}$ and gives their product $P = a\cdot b$ to Gauss and their sum $S = a+b$ to Euler. The two mathematicians are not allowed to communicate, but they can speak out loud their thoughts.

The conversation proceeds as follows:
\begin{enumerate}
\item Gauss: \emph{I don't know the numbers.}
\item Euler: \emph{I knew that you didn't know.}
\item Gauss: \emph{Now I know the numbers.}
\item Euler: \emph{Now I know them too.}
\end{enumerate}

\section{Solution}
Gauss does not know the solution. This means that the factorization of $P$ into two numbers is not unique. In addition, Euler knew that Gauss did not knew, which means for each pair of numbers $(a',b')$ which sum up to S, the product $P' = a'\cdot b'$ cannot be factorized uniquely into to factors $a$ and $b$.

By going through all possible sums $S \in \{4,\ldots,198\}$. We find the following possible values for $S$:

11, 17, 23, 27, 29, 35, 37, 41, 47, 53

TO BE FINISHED
\end{document}
